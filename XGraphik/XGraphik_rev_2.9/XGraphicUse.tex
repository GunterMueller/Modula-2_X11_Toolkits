\documentstyle[german,din-a4,listing,twoside,abkuerzungen]{article}
%\makeindex
\pagestyle{headings}

\title{Programmierung mit der XGraphic-Bibliothek}
\author{Alexander Sch"olling \and
Stefan Wohlfeil}
\date{August 1996}

\begin{document}
% Auf das Inhaltsverzeichnis verzichten wir besser.
% 15.3 Stefan
% \tableofcontents
% \pagebreak
\maketitle


\newcommand{\prochead}[1]{{{#1}\index{#1}}}
\newcommand{\syntax}[1]{\item[Syntax:] {\tt #1}}
\newcommand{\param}[1]{\item[Parameter:] {#1}}
\newcommand{\func}[1]{\item[Funktionalit"at:] {#1}}
\newenvironment{commanddesc}{%
	\begin{description}
}{%
	\end{description}
}




\section{Einf"uhrung}

 Die Programmierung unter einem graphischen Fenstersystem (wie \zB\  
 X-Windows) unterscheidet sich grundlegend von der Programmierung auf
 zeilenorientierten Terminals.
 Dort hat in der Regel das Programm den Benutzer gef"uhrt. Brauchte
 das Programm verschiedene Eingaben vom Benutzer, so hat es diese
 nacheinander, \zB\  durch Men"us abgefragt. Ein Benutzer konnte also
 immer nur eine bestimmte Eingabe machen.

 Ein typisches Programmst"uck sah wie folgt aus:
\begin{verbatim}
 REPEAT
    ZeigeMenue1();
    Taste := LeseTastatur()
 UNTIL ErlaubtInMenue1 (Taste);
 REPEAT
    ZeigeMenue2();
    Taste := LeseTastatur()
 UNTIL ErlaubtInMenue2 (Taste);
 ...
\end{verbatim}

 Heutzutage m"ochte ein Benutzer sich aber nicht vom Programm vorschreiben
 lassen, wann er welche Eingaben machen mu"s. Das Programm soll auf die
 Eingaben des Benutzers reagieren. Aus Sicht des Programms treten also
 st"andig Ereignisse auf, auf die das Programm reagieren mu"s. M"ogliche
 Ereignisse sind:

\begin{itemize}
   \item Taste gedr"uckt bzw. losgelassen
   \item rechter/mittlerer/linker Mausknopf gedr"uckt bzw. losgelassen
   \item Mauszeiger ist in einen bestimmten Bildschirmbereich eingetreten
   \item ...
\end{itemize}

 Au"serdem hat man heute in der Regel mehrere Programme gleichzeitig
 auf einem Computer laufen. Daher m"ussen die auftretenden Ereignisse
 dem richtigen Programm zugeordnet werden. Dr"uckt der Benutzer \zB\ 
 eine Taste, so mu"s entschieden werden welches Programm auf diesen
 Tastendruck reagieren soll.
 
 Ein typisches Programmst"uck sah wie folgt aus:
\begin{verbatim}
 REPEAT
    Ereignis := HoleEreignis();
    BearbeiteEreignis (Ereignis);
 UNTIL fertig;

 PROCEDURE BearbeiteEreignis ( Ereignis : ... );
 BEGIN
    CASE Ereignis OF

       (* lange Liste aller moeglichen Ereignisse *)

    END; (* CASE *)
 END BearbeiteEreignis;
\end{verbatim}

 Um sich die Programmierung zu erleichtern, gibt es nun verschiedene
 Bibliotheken, die einige, immer gleich zu behandelnde Ereignisse
 bearbeiten und ansonsten die Ereignisse automatisch an das richtige
 Programm weiterleiten. Als Programmierer mu"s man nur noch angeben,
 welche der selbstgeschriebenen Funktionen bei einem bestimmten
 Ereignis aufgerufen werden soll.

 Der typische Ablauf dabei ist wie folgt: Zun"achst erzeugt der
 Programmierer die Objekte (Fenster, Menues, Scrollbalken, Kn"opfe, etc.)
 aus denen die Oberfl"ache bestehen soll. Anschliessend wird eine
 Prozedur aufgerufen, die die eintretenden Ereignisse "uberwacht und
 die jeweiligen Funktionen, die der Programmierer bei der Erzeugung
 der Objekte angegeben hat, aufruft.

 Ein typisches Programmst"uck sieht wie folgt aus:
\begin{verbatim}
 PROCEDURE BearbeiteKnopf ( ... (* Parameter siehe jeweils im Handbuch *) );
 BEGIN
    WriteString("Der Knopf wurde gedrueckt!");
 END BearbeiteKnopf;

 BEGIN
 ...
 Erzeuge_Objekt( Knopf, 
     KnopfText,  "Hier druecken!",
     KnopfKoordinaten,   100, 100,
     KnopfEventHandler,  BearbeiteKnopf, (* Hier wird als Parameter die   *)
                                         (* Adresse einer Prozedur ueber- *)
                                         (* geben.                        *)
     ENDE );

 Hauptschleife();
 END.
\end{verbatim}

 In der Prozedur Hauptschleife, die vom System zur Verf"ugung gestellt wird,
 werden alle auftretenden Events bearbeitet und bei Bedarf die vom
 Programmierer geschriebenen Funktionen (BearbeiteKnopf) aufgerufen.
 Mehr zu diesem Thema k"onnen Sie \ua\ im Kurs 1817 {\em Programmierung
 graphischer Benutzerschnittstellen} nachlesen.

 Welche Objekte vorhanden sind, welche Attribute (\zB\  Koordinaten) ein
 bestimmtes Objekt haben kann, wie die Parameterliste der zu
 programmierenden Call-back Funktionen (BearbeiteKnopf) genau aussieht
 und viele weitere Details mu"s der Programmierer den Handb"uchern der
 Bibliothek entnehmen. Der typische Umfang dieser Handb"ucher liegt
 bei ca.\ 600 Seiten. Damit Sie sich im Praktikum nun nicht mit allen
 Details befassen m"ussen, habe wir eine kleine Bibliothek mit den
 wichtigsten Funktionen f"ur Sie erstellt.
 


%
% Noch einen Abschnitt zu Toolkits wie xview einfuegen!
% Werde ich (Stefan) noch schreiben.
%
% 14.3.94 Erledigt.
%

Die Bibliothek XGraphic erleichtert Ihnen die Programmierung von unter
X Windows ablaufenden Programmen, indem sie etliche Routinearbeiten
bereits erledigt. Dazu geh"ort zum Beispiel der vordefinierte Button
QUIT mit der dazugeh"origen Funktion; dadurch brauchen Sie sich nicht
mehr darum k"ummern, da"s ein Benutzer Ihr Programm beenden kann. Auch
wird die Programmierung von Men"us auf die Verwendung von drei Befehlen
mit wenigen Parametern reduziert. Die Bibliothek stellt Ihnen weiterhin
einfache Befehle zur Verf"ugung, um herauszufinden, wo sich die Maus
befindet oder welche Taste gedr"uckt wurde. In Kapitel \ref{Beispiel}
sind zwei Beispielprogramme abgedruckt. Das Erste ist das
k"urzestm"ogliche Beispielprogramm, es tut nichts au"ser ein Fenster
auf den Bildschirm bringen. In diesem Fenster passiert aber nichts. Das
zweite Beispielprogramm erledigt bereits eine einfache
Benutzerkommunikation mit zwei einfachen Men"us und einem einfachen
Malteil.





\section{Namenskonventionen}
Alle Funktionsnamen aus der Library beginnen in jedem Fall mit den
Gro"sbuchstaben XG, gefolgt von entweder einem Unterstrich oder einem
weiteren Gro"sbuchstaben und einem Unterstrich. Dieses K"urzel gibt die
Zugeh"origkeit der Funktion zu einer der folgenden Kategorien an:
\begin{itemize}\index{Funktionsgruppen}
   \item[XG]{Allgemeine Funktionen, Initialisierung des Systems und
       Start der Hauptschleife. Siehe Kapitel: \ref{XG}.}
   \item[XGM]{Men"ufunktionen, Definition und "Anderung von
       Men"us. Siehe Kapitel: \ref{XGM}.}
   \item[XGF]{Fu"szeilen-Funktionen, Textausgabe in die
       Fu"szeile des Fensters. Siehe Kapitel: \ref{XGF}.}
   \item[XGN]{Ein-/Ausgabe-Funktionen, Funktionen f"ur
       Mitteilungsfenster und Eingabefenster. Siehe Kapitel: \ref{XGN}.}
   \item[XGD]{Malfunktionen, Funktionen zum Zeichnen
       und Schreiben in den Malbereich des Fensters. Siehe Kapitel: \ref{XGD}.}
   \item[XGE]{Event-Behandlung, Funktionen zur Aufschl"ussellung
       von Events. Kapitel \ref{XGE}.}
\end{itemize}





\section{Funktionsdefinitionen}
In diesem Abschnitt stellen wir Ihnen die einzelnen Funktionen der
Library vor. Zu jeder Funktion geben wir eine Erkl"arung zu den
Parametern der Funktion, sowie eine kurze Beschreibung was die
Funktion macht. Die genauen Namen der Funktion, die Typen der
Parameter, die Reihenfolge der Parameter, etc. entnehmen Sie dem
Definitionsmodul zu unserer Library das auf dem Rechner zur
Verf�gung steht.
%
% evtl. einen Verweis einfuegen, wo der Definition Module zu
% finden ist. Das muss nicht hier im Dokument sein.
%
% abgehakt, Alexander, 9.3.94



\subsection{Allgemeine Funktionen\label{XG}}

\subsubsection{\prochead{Init}}
\begin{commanddesc}
\syntax{XG\_Init (AnzahlParameter, ParameterListe,
PROC(RepaintCallback), PROC(Event\-Callback));}
\param{"Ubergeben werden die Anzahl und eine Tabelle
der Kommandozeilen-Parameter sowie die Addressen der
beiden ben"otigten Callback-Funktionen.}
\func{Das Programmfenster wird auf den Bildschirm gebracht
und initialisiert.}
\end{commanddesc}

\subsubsection{\prochead{StartMainLoop}}
\begin{commanddesc}
\syntax{XG\_StartMainLoop();}
\param{Keine.}
\func{Die Hauptschleife des Systems wird gestartet. Die Kontrolle
liegt jetzt beim System und nur wenn ein Ereignis eintritt,
wird eine Callback-Funktion aufgerufen.}
\end{commanddesc}



\subsection{Men"u-Definition\label{XGM}}
Die XGraphic-Bibliothek erlaubt die Erstellung einer
Men�-Leiste, die am oberen Rand des Fensters angezeigt
wird.
Die Men�-Leiste besteht aus Men�s, die nacheinander
erstellt werden und in ihrer Erstellungsreihenfolge
von links nach rechts im Fenster angezeigt werden.
Klick man mit der Maus auf ein Men�, so werden die
einzelnen Men�-Punkte sichtbar.
Die Men�-Punkte werden in ihrer Erstellungsreihenfolge
von oben nach unten angezeigt.
Klick man auf einen Men�-Punkt, so wird der Callback
des Men�s aufgerufen und bekommt als Parameter die
Nummer des angeklickten Men�-Punkts �bergeben.

\subsubsection{\prochead{NewMenu}}
\begin{commanddesc}
\syntax{IntVar := XGM\_NewMenu (LabelString);}
\param{Eine Zeichenkette (Men"u-Label).}
\func{Einleiten der Definition eines neuen Men"us. Es wird ein Handle
auf das Men"u zu\-r"uck\-ge\-liefert, mit dessen Hilfe sp"ater im Programm
einzelne Men"u-Punkte deaktiviert und wieder aktiviert werden k"onnen.}
\end{commanddesc}

\subsubsection{\prochead{AppendItem}}
\begin{commanddesc}
\syntax{XGM\_AppendItem (MenuHandle, ItemString);}
\param{Eine Zeichenkette sowie der Handle aus dem Aufruf von NewMenu.}
\func{Ein Men"u-Punkt wird an das Men"u angeh"angt.}
\end{commanddesc}

\subsubsection{\prochead{CreateMenu}}
\begin{commanddesc}
\syntax{XGN\_CreateMenu (HMenuHandle, PROC(MenuCallbackProc));}
\param{Addresse der Men"u-Callback-Funktion.
Handle des Men"us.}
\func{Diese Funktion schlie"st die Definition eines Men"us ab.
Erst mit diesem Aufruf wird das Men"u wirklich erzeugt und in
das Fenster eingebaut!}
\end{commanddesc}

\subsubsection{\prochead{SetItemActivity}}
\begin{commanddesc}
\syntax{XGM\_SetItemActivity (MenuHandle, ItemNumber, StatusFlag);}
\param{"Ubergeben werden der Men"uhandle, die laufende Nummer
des zu bearbeitenden Eintrags (Numerierung beginnt mit eins)
sowie der neue Status (Bool'scher Wert, TRUE entspricht ausw"ahlbar,
FALSE inaktiv).}
\func{Mit SetItemActivity k"onnen einzelne Punkte von Men"us
deaktiviert bzw.\ reaktiviert werden.}
\end{commanddesc}



\subsection{Fu"szeilen bearbeiten\label{XGF}}
Am unteren Bildschirmrand k�nnen in einem eigenen Bereich
einzeilige Zeichenketten ausgegeben werden.
Die Zeichenketten schlie�en b�ndig mit dem �u�eren
Rand ab.
Sind die Zeichenketten zusammen breiter als der Bereich,
so �berlappen sie in der Mitte.

\subsubsection{\prochead{SetLeft}}
\begin{commanddesc}
\syntax{XGF\_SetLeft (StringConst);}
\param{Eine Zeichenkette (anzuzeigender String).}
\func{Der "ubergebene String wird in der linken Fu"szeile
angezeigt und "uberschreibt damit den bisher dort angezeigten
Text.}
\end{commanddesc}

\subsubsection{\prochead{SetRight}}
\begin{commanddesc}
\syntax{XGM\_SetRight (StringConst);}
\param{Eine Zeichenkette (anzuzeigender String).}
\func{Der "ubergebene String wird in der rechten Fu"szeile angezeigt und
"uberschreibt damit den bisher dort angezeigten Text.}
\end{commanddesc}

\subsubsection{\prochead{SetDefault}}
\begin{commanddesc}
\syntax{XGM\_SetDefault();}
\param{Keine.}
\func{Der Text in den Fu"szeilen wird durch den Standard-Text ersetzt
(Links: Copyright-Meldung; Rechts: Versions-Nummer).}
\end{commanddesc}



\subsection{Ein-/Ausgabe-Fenster\label{XGN}}
Damit bei der Eingabe einer Zahl oder einer Zeichenkette nicht
jeder einzelne Tastendruck von einem selbstprogrammierten
Event-Handler bearbeitet werden mu�, gibt es Hilfsfunktionen,
die die bequemere Eingabe von Zahlen, Zeichenketten,
Dateinamen und Ja/Nein-Entscheidungen erlauben.

\subsubsection{\prochead{InputInt}}
\begin{commanddesc}
\syntax{XGN\_InputInt (PromptString, InitialValue,
MinValue, MaxValue, NumberOfTicks);}
\param{Eine Zeichenkette (Prompt), Vorgabewert, Untergrenze,
Obergrenze, Anzahl Unterteilungen.}
\func{Es wird ein Fenster angezeigt, in dem der Benutzer durch Verschieben
des Sliders eine Zahl im angegebenen Bereich eingeben kann. Es werden
keine Events generiert, bevor der Benutzer eine Zahl eingegeben hat.
Das Fenster enth"alt zwei Buttons, die mit OK und CANCEL beschriftet
sind. Egal welcher Button gedr"uckt wird, mit der Funktion GetLastInteger
wird in jedem Fall der gew"ahlte Wert geliefert. Weitere Erl"auterungen
unter der Funktion GetlastInteger.}
\end{commanddesc}


\subsubsection{\prochead{GetlastInteger}}
\begin{commanddesc}
\syntax{BoolVar := XGN\_GetLastInteger (IntVar);}
\param{Eine INTEGER-Variable.}
\func{Mit dieser Funktion kann der letzte eingegebene Wert erfragt werden.
Der R"uckgabewert ist TRUE, wenn der Benutzer den OK-Button dr"uckte,
FALSE sonst.}
\end{commanddesc}

\subsubsection{\prochead{InputString}}
\begin{commanddesc}
\syntax{XGN\_InputString (LabelString);}
\param{Eine Zeichenkette (Prompt).}
\func{Wie auch InputInt bringt InputString ein Fenster auf den Bildschirm, in
dem der Benutzer einen Text eingeben kann. Die Generierung von Events
ist blockiert, bis der Benutzer einen der beiden Buttons anklickt.}
\end{commanddesc}

\subsubsection{\prochead{GetLastString}}
\begin{commanddesc}
\syntax{BoolVar := XGN\_GetLastString (StringVar);}
\param{Eine String-Variable.}
\func{Wenn der Benutzer die Eingabe durch den Button OK abgeschlo"sen hat, so
wird TRUE als R"uckgabewert geliefert. Wurde der CANCEL-Button
angeklickt, so wird FALSE zur"uckgegeben und als Text wird "CANCELLED"
geliefert.}
\end{commanddesc}

\subsubsection{\prochead{ShowNotice}}
\begin{commanddesc}
\syntax{XGN\_ShowNotice (StringConst);}
\param{Eine Zeichenkette (Auszugebender Text).}
\func{Die Ausf"uhrung des Programms wird solange angehalten, bis der Benutzer
"uber den OK-Button die Anzeige quittiert hat. Der Text kann Newline-Zeichen
(ASCII 13) enthalten, um einen Zeilenumbruch des Textes zu erzwingen.}
\end{commanddesc}

\subsubsection{\prochead{FileSelector}}
\begin{commanddesc}
\syntax{XGN\_FileSelector (MaskString);}
\param{Eine Zeichenkette (regul"arer Ausdruck), die die
anzuzeigenden Dateien angibt.}
\func{Es wird eine Fileselector-Box auf den Bildschirm gebracht, die
den Inhalt des aktuellen Verzeichnisses alphabetisch sortiert
(Verzeichnisse vor Dateien) anzeigt.}
\end{commanddesc}

\subsubsection{\prochead{GetFileName}}
\begin{commanddesc}
\syntax{BoolVar := XGN\_GetFileName (StringVar);}
\param{Eine Stringvariable.}
\func{Wie auch bei GetLastInteger und GetLastString wird hier
der zuletzt eingegebene Wert (Dateiname inkl. Pfad) sowie der
gedr"uckte Button (TRUE entspricht OPEN) geliefert.}
\end{commanddesc}

\subsubsection{\prochead{YesOrNo}}
\begin{commanddesc}
\syntax{BoolVar := XGN\_YesOrNo (StringConst);}
\param{Zeichenkette.}
\func{Diese Funktion gibt eine Notice aus (siehe ShowNotice),
die jedoch zwei Buttons enth"alt (YES und NO). TRUE wird geliefert,
wenn YES gedr"uckt wurde, FALSE sonst.}
\end{commanddesc}



\subsection{Malfunktionen\label{XGD}}
%
% Hier koennten wir auch kurz etwas zum verwendeten Koordinatensystem
% sagen. Wo liegt (0,0) und in welche Richtung zeigen steigende x bzw.
% y Werte? Welches ist der maximale x bzw. y Wert?
%
Zwischen den Men�s und der Fu�zeile liegt der Zeichenbereich.
Hierin k�nnen Grafiken und Texte ausgegeben werden.
Der Wertebereich der Koordinaten liegt zwischen $(0, 0)$
und $(9999, 9999)$, der Ursprung liegt in der oberen linken Ecke.
Die X-Koordinaten wachsen nach rechts, die Y-Koordinaten nach
unten.
Bei allen Funktionen wird zuerst die X-Koordinate angegeben.
Die Farben sind als Aufz"ahlungstyp im Definitions-
Modul abgelegt.

\subsubsection{\prochead{NoClipping}}
\begin{commanddesc}
\syntax{XGD\_NoClipping;}
\func{Schaltet das Clipping f"ur alle Malfunktionen auf Fensterdimension
zur"uck. Die ist gleichzeitig auf \em{Voreinstellung}.}
\end{commanddesc}

\subsubsection{\prochead{SetClipping}}
\begin{commanddesc}
\syntax{XGD\_SetClipping (pointX, pointY, width, height);}
\param{Die Koordinaten der oberen, linken Ecke des Clipping-Rechtecks
sowie dessen Breite und H"ohe.}
\func{Setzt das Clipping-Rechteck f"ur alle Malfunktionen.}
\end{commanddesc}

\subsubsection{\prochead{BeginPoly}}
\begin{commanddesc}
\syntax{XGD\_BeginPoly (MaxN);}
\param{"Ubergeben wird die gew"unschte Maximalzahl von Polygonpunkten.}
\func{Leitet die Definition eines zu malenden Polygons oder Linienzuges ein.}
\end{commanddesc}

\subsubsection{\prochead{AddPolyPoint}}
\begin{commanddesc}
\syntax{XGD\_AddPolyPoint (px, py);}
\param{Die Koordinaten des neuen Polygonpunktes. Wenn mehr Aufrufe
erfolgen, als Polygonpunkte bei BeginPoly angegeben waren, so werden
die zus"atzlichen Punkte ignoriert. Um ein Polygon zu schliessen, wird
{\em nicht} der Startpunkt nocheinmal als letzter Punkt angegeben!}
\func{Anf"ugen eines Punktes an das mit BeginPoly eingeleitete Polygon.}
\end{commanddesc}

\subsubsection{\prochead{DrawPoly}}
\begin{commanddesc}
\syntax{XGD\_DrawPoly (Closed, Filled, Width, Color);}
\param{Zwei Bool'sche Werte (Closed und Filled), die angeben, ob das
Polygon geschlo"sen bzw. gef"ullt ist. Die Breite, in der die Linien
gezeichnet werden sowie die Farbe.}
\func{Schlie"st die Definition eines Polygons oder eines Linienzuges ab
und malt ihn.  Wenn $Closed = \mbox{TRUE}$, so wird eine Linie vom
letzten zum ersten Punkt gezeichnet. Wenn $Filled = \mbox{TRUE}$, so
wird das Polygon in der angegebenen Farbe gef"ullt. Es wird dann
nat"urlich auch geschlossen.}
\end{commanddesc}

\subsubsection{\prochead{DrawLine}}
\begin{commanddesc}
\syntax{XGD\_DrawLine (StartX, StartY, EndX, EndY, Breite, Farbe);}
\param{F"unf ganze Zahlen (Anfangs- und Endkoordinaten, Liniendicke).
Element von tColor (Farbe).}
\func{Die Funktion zeichnet eine Linie vom Start- zum Endpunkt in der
angegebenen Dicke und Farbe.}
\end{commanddesc}

\subsubsection{\prochead{DrawCircle}}
\begin{commanddesc}
\syntax{XGD\_DrawCircle (CenterX, CenterY, Radius, Filled, Color);}
\param{Vier ganze Zahlen (Mittelpunkts-Koordinaten, Radius, Farbe).
Ein Wahrheitswert, der angibt, ob der Kreis gef"ullft werden soll.}
\func{Die Funktion zeichnet einen Kreis mit dem gew"unschten Radius um
den angegebenen Mittelpunkt in der angegebenen Farbe. Wenn Filled TRUE
ist, so wird der Kreis in der entsprechenden Farbe ausgef"ullt.}
\end{commanddesc}

\subsubsection{\prochead{DrawString}}
\begin{commanddesc}
\syntax{XGD\_DrawString (PosX, PosY, Text, Color);}
\param{Zwei ganze Zahlen (Koordinaten), die Farbe und eine
Zeichenkette.}
\func{Die "ubergebene Zeichenkette wird in der angegebenen Farbe
ausgegeben.  Die "ubergebenen Koordinaten bezeichnen die linke untere
Ecke der Bounding Box!}
\end{commanddesc}

\subsubsection{\prochead{GetStringExtents}}
\begin{commanddesc}
\syntax{XGD\_GetStringExtents (Text, Breite, Hoehe);}
\param{Ein String (Text), sowie zwei INTEGER-Variablen (Breite, Hoehe).}
\func{In den INTEGER-Variablen werden die Abmessungen des Strings (in
Pixeln) zur"uckgegeben.}
\end{commanddesc}

\subsubsection{\prochead{Clear}}
\begin{commanddesc}
\syntax{XGD\_Clear();}
\param{Keine.}
\func{Der Malbereich des Programmfensters wird gel"oscht.}
\end{commanddesc}

\subsubsection{\prochead{DrawArrowLine}}
\begin{commanddesc}
\syntax{XGD\_DrawArrowLine(StartX, StartY, EndX, EndY, Width, Color);}
\param{Erl"auterungen siehe DrawLine.}
\func{Es wird eine Linie zwischen den angegebenen Punkten gemalt. Bei
60 Prozent der L"ange befindet sich eine Pfeilspitze.}
\end{commanddesc}

\subsubsection{\prochead{Flush}}
\begin{commanddesc}
\syntax{XGD\_Flush();}
\param{Keine.}
\func{Der Windowmanager wird gezwungen, ein Repaint durchzuf"uhren
(wird ben"otigt, wenn \zB\  als Folge eines animierten Ablaufs eines
Malvorgangs ein Repaint ben"otigt w"urde, jedoch nicht automatisch
ausgef"uhrt wird.}
\end{commanddesc}



\subsection{Event-Behandlung\label{XGE}}
Wenn die Maus geklickt oder eine Taste gedr�ckt wurde,
ruft die Hauptschleife den Event-Handler auf.
Dieser mu� jetzt erfahren, welcher Art das Ereignis denn
nun konkret war, wo die Maus positioniert war, als das
Ereignis eingetreten ist, usw.

\subsubsection{\prochead{Button\_L}}
\begin{commanddesc}
\syntax{BoolVar := XGE\_Button\_L();}
\param{Keine.}
\func{Diese Funktion gibt zur"uck, ob der Event durch die linke
Maustaste ausgel"ost wurde (R"uckgabewert TRUE) oder nicht.}
\end{commanddesc}

\subsubsection{\prochead{Button\_M}}
\begin{commanddesc}
\syntax{BoolVar := XGE\_Button\_M();}
\param{Keine.}
\func{Diese Funktion arbeitet wie Button\_L, gibt jedoch an, ob die
mittlere Maustaste den Event ausgel"ost hat.}
\end{commanddesc}

\subsubsection{\prochead{Button\_R}}
\begin{commanddesc}
\syntax{BoolVar := XGE\_Button\_R();}
\param{Keine.}
\func{Wie Button\_L, jedoch f"ur die rechte Maustaste.}
\end{commanddesc}

\subsubsection{\prochead{X\_Position}}
\begin{commanddesc}
\syntax{IntVar := XGE\_X\_Position();}
\param{Keine.}
\func{Die X-Koordinate der Maus zum Zeitpunkt des Eintretens des Events.}
\end{commanddesc}

\subsubsection{\prochead{Y\_Position}}
\begin{commanddesc}
\syntax{IntVar := XGE\_Y\_Position();}
\param{Keine.}
\func{Wie X\_Position, jedoch f"ur die Y-Koordinate.}
\end{commanddesc}

\subsubsection{\prochead{KeyCode}}
\begin{commanddesc}
\syntax{CharVar := XGE\_KeyCode();}
\param{Keine.}
\func{Als R"uckgabewert wird der ASCII-Code des Zeichens geliefert, das
den aktuellen Event ausgel"ost hat, beziehungsweise das Null-Zeichen,
wenn keine Taste f"ur den Event verantwortlich war.}
\end{commanddesc}




\section{Beispiele\label{Beispiel}}




\subsection{Einfachstes Programm}

\listing{simple.m2i}

Dieses einfache Beispielprogramm bringt lediglich ein Fenster auf den
Bildschirm, in dem nicht gemalt wird. Die Funktionen {\em Repaint} und
{\em Event} sind leere Funktionen, die jedoch trotzdem vorhanden sein
m"ussen. Durch den Befehl {\em PROC(Repaint)} wird erreicht, da"s an
dieser Stelle nicht die Funktion Repaint aufgerufen, sondern nur ihre
Addresse eingesetzt wird.  (Analog f"ur Event.)




\subsection{Ein Gr"o"seres Beispiel}
\listing{greatersample.m2i}

In diesem Beispiel wird zus"atzlich ein Men"u eingef"ugt, das bei Auswahl des
zweiten Men"u-Punktes eine Nachricht auf den Bildschirm bringt. Au"serdem
reagiert das Programm auf Mausclicks innerhalb des Malbereichs (der gro"sen
Fl"ache unterhalb der Men"uzeile), indem es um den Punkt, an dem die linke
Maustaste gedr"uckt wurde, einen Kreis zeichnet.
    
%
% Kurze Erklaerung zu diesem Beispiel
%
% erledigt - Alexander


\section{Definitionsmodul}

\listing{xgraphic.m2d}

\end{document}
