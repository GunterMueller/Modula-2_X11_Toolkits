\documentstyle[german,din-a4,listing,abkuerzungen]{article}
%\makeindex
\pagestyle{headings}

\title{Installation der XGraphic-Bibliothek}
\author{Alexander Sch"olling \and
Stefan Wohlfeil}

\begin{document}
% Auf das Inhaltsverzeichnis verzichten wir besser.
% 15.3 Stefan
% \tableofcontents
% \pagebreak
\maketitle

\newcommand{\prochead}[1]{{{#1}\index{#1}}}
\newcommand{\syntax}[1]{\item[Syntax:] {\tt #1}}
\newcommand{\param}[1]{\item[Parameter:] {#1}}
\newcommand{\func}[1]{\item[Funktionalit"at:] {#1}}
\newenvironment{commanddesc}{%
	\begin{description}
}{%
	\end{description}
}


\section{Systemanforderungen}

\subsection{Hardware}
Um XGraphic benutzen zu k"onnen, ben"otigen Sie ein lauff"ahiges
Linux-System mit konfiguriertem X-Server. Auf der Linux-Partition
m"ussen noch ca. 2.5 - 3 MB f"ur den Compiler (falls noch nicht
vorhanden) sowie ca. 4 MB f"ur die Library frei sein.

\subsection{Software}
Verwenden k"onnen Sie Linux Kernel Versionen ab 0.99.13 mit XFree86
ab Version 2.0. Diese Konfiguration wurde allerdings von uns nicht
mit der neuesten XGraphic-Version getestet, bei uns l"auft Kernel
1.2.13 mit XFree86 3.1.2. Wir verwenden die MOCKA-Compilerversion 9502
und GCC Version 2.7.2.


\section{Installation}

\subsection{Allgemeines}
Da sich die Installationsprogramme der zur Zeit gebr"auchlichen
Linuxdistributionen so weit unterscheiden und wir nicht alles testen
k"onnen, mu"s die Installation der Library von Hand ausgef"uhrt werden.
Dabei k"onnen Sie den Abschnitt 1 - Installation des Compilers "uberspringen,
wenn Sie den MOCKA-Compiler bereits installiert haben\footnote{Dies ist
der Fall, wenn Sie in einer Shell mit 'mocka' oder 'mc'
eine Meldung 'Mocka XXXX' erhalten. Verlassen k"onnen Sie MOCKA durch Eingabe
von 'q' und Enter.} oder das Praktikum in C++ machen wollen.
In jedem Fall m"ussen Sie die Installation als Superuser (root) durchf"uhren.
Wenn Sie also noch nicht als root eingeloggt sind, so loggen Sie sich jetzt
bitte aus und melden sich als root wieder an.

\subsection{Schritt f"ur Schritt}
Bei allen Kommandos gilt: Die umschlie"senden Hochkommata (') d"urfen nicht
mit eingegeben werden!! Je nach verwendeter Version der MTools k"onnen die
Dateinamen auch durchg"angig gro"sgeschrieben sein; dann ist bei den
Kopieroperationen der Zieldateiname unbedingt mit anzugeben; andernfalls
kann dieser weggelassen werden. Ebenso sind dann die Namen der Archive und
der zu kopierenden Dateien durchg"angig gro"s zu schreiben.

\subsubsection{Abschnitt 0 - Vorbereitungen}
\begin{enumerate}
\item{Diskette einlegen}
\item{Wechsel nach /tmp: 'cd /tmp'}
\item{Erstellen eines Hilfsverzeichnisses: 'mkdir sopra'}
\item{Wechsel in dieses Verzeichnis: 'cd sopra'}
\item{Kopieren der Dateien: 'mcopy a:*.* .'}
\end{enumerate}

\subsubsection{Abschnitt 1 - Installation des Compilers}
\begin{enumerate}
\item{Wechsel nach /usr/local: 'cd /usr/local'}
\item{Entpacken des Archivs: 'tar -xzf /tmp/sopra/mocka.tgz'}
\item{Installieren:}
\begin{enumerate}
\item{Wechsel nach mocka: 'cd mocka'}
\item{Installation: 'make install'}
\item{Wechsel nach /usr/local/bin: 'cd /usr/local/bin'}
\item{Anlegen eine Links von mc nach mocka: 'ln -s mc mocka'}
\end{enumerate}
\end{enumerate}

\subsubsection{Abschnitt 2 - Installation von XGraphic}
\begin{enumerate}
\item{Wechsel nach /tmp: 'cd /tmp/sopra'}
\item{Kopieren des Modula-Pr"aprozessors: 'cp m2pp /usr/local/bin/m2pp'}
\item{Kopieren des ge"anderten Linkskriptes: 'cp link /usr/local/mocka/sys/link'}
\item{Setzen der Umgebungsvariable OSNAME: 'export OSNAME=Linux'}
\item{Entpacken des Archivs: 'tar -xzf ./xgraphic.tgz'}
\item{"Ubersetzen und installieren:}
\begin{itemize}
\item{F"ur MOCKA: 'make mockainstall'}
\item{F"ur C/C++: 'make cinstall'}
\end{itemize}
\end{enumerate}

\subsubsection{Abschnitt 3 - Installation von nsgmls}
Nsgmls ist das Hilfsprogramm, mit dem Sie aus HTML-Dateien die f"ur
Ihr Programm ben"otigten Eingabedateien generieren werden.

\begin{enumerate}
\item{Entpacken der Programme: 'tar -xzf nsgmls.tgz'}
\item{Installation der Programme: './instsgml.sh'}
\item{Wechsel nach /usr/local: 'cd /usr/local'}
\item{Anlegen des Katalog-Verzeichnisses: 'mkdir sgml'}
\item{Wechsel in dieses Verzeichnis: 'cd sgml'}
\item{Entpacken der Hilfsdateien: 'tar -xzf /tmp/sopra/sgml.tgz'}
\item{Wechsel nach /etc: 'cd /etc'}
\item{Editieren von profile: 'vi profile', es m"ussen folgende Zeilen
eingef"ugt werden:}
\begin{itemize}
\item{export SGML\_SEARCH\_PATH=/usr/local/sgml;}
\item{export SGML\_CATALOG\_FILES=/usr/local/sgml/catalog;}
\item{export OSNAME=Linux;}
\end{itemize}
\item{Aufr"aumen: 'rm -rf /tmp/sopra'}
\end{enumerate}

\end{document}
