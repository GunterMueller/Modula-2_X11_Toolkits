\documentstyle[german,din-a4,time]{article}
\parindent0pt
\parskip0.2\baselineskip

%
% In der endgueltigen Version den Header auskommentieren!
%
%\pagestyle{myheadings}
%\markright{Installationsanleitung \today\ \thetime}

\newcommand{\taste}[1]{[#1]}
\newcommand{\prog}[1]{{\bf #1}}
\newcommand{\eing}[1]{{\tt #1}}
\newcommand{\datei}[1]{{\tt #1}}

\begin{document}


\title{Installationsanleitung der LINUX-Version PROPRA\\
f�r Herrn Leckebusch}
\author{Alexander Sch�lling, Stefan Wohlfeil}
\date{April 1994}
% \date{\today\ \thetime }
\maketitle

\section{Einleitung}
Dieser Text ist ein Auszug aus einem Text, den wir f�r die Studenten,
die zu Hause selbst LINUX installieren wollen, geschrieben haben. Da
wir den Studenten einige Sachen empfehlen, die wir auf den Leih-PCs nicht
machen wollen, sind einige Schritte von uns noch nicht ausprobiert worden.

Wenn es daher Probleme geben sollte, wenden Sie sich bitte direkt an
mich (Tel.: 4371). Ansonsten viel Spa�...

\section{Installation von LINUX auf Rechenzentrum PCs}
Vor Beginn der Installation eine leere, formatierte 5.25''-Diskette
bereithalten.
Bei der Installation sollen die folgenden Schritte durchgef�hrt werden:
\begin{itemize}
\item L�schen aller alten Partitionen Ihrer Festplatte. (Mit dem 
	DOS-Programm \prog{fdisk})
\item Anlegen einer neuen, kleinen MS-DOS Partition. (Mit dem 
	DOS-Programm \prog{fdisk})
\item Anlegen einer neuen SWAP Partition\footnote{
In dieser Partition lagert das Betriebssystem Teile des Hauptspeichers aus,
die gerade nicht ben�tigt werden.} f�r LINUX. (Mit dem 
	LINUX-Programm \prog{fdisk})
\item Anlegen einer neuen, gro�en LINUX Partition. (Mit dem 
	LINUX-Programm \prog{fdisk})
\item Installation von LINUX auf der LINUX Partition.
\item Installation des LILO Bootmanagers
\end{itemize}


\section{L�schen und einrichten der DOS-Partition}

Da wir nicht aus noch Drucker unter LINUX unterst�tzen wollen, soll eine
kleine DOS-Partition auf der Platte bleiben. Wir dachten an eine Gr��e
von 50 MB. Wir installieren dort nur das Betriebssystem, damit die
Studenten Dateien unter DOS drucken k�nnen.

Platte dann formatieren (\eing{format /s c:}) und die paar DOS Dateien
auf die Platte kopieren.


\section{Installation von LINUX}

Der Rechner wird von unserer LINUX Bootdiskette gebootet.
Danach erscheint auf dem Bildschirm folgende Meldung:

\begin{verbatim}
Please remove the boot Kernel disk from your floppy drive,
insert a disk to be loaded into the ramdisk, and press
[enter] to continue.
\end{verbatim}

Jetzt Diskette {\bf color 144} in Laufwerk B: legen.

Nachdem der Rechner jetzt l�uft, kann man sich ohne Passwort als
\eing{root} anmelden und die Platte weiter Partitionieren.
Als erstes sollte die Tastatur auf \eing{German} umgestellt werden. Dazu den
folgenden Befehl eingeben:

\begin{verbatim}
# cd /usr/lib/kbd/keytables
# gzip -d <gr-lat1.map.gz | loadkeys
\end{verbatim}

Als n�chstes wird \eing{fdisk} gestartet. Damit soll dann eine
16 MB gro�e SWAP-Partition angelegt werden. (Der Id-Code der
SWAP-Partition mu� extra gesetzt werden!) Der Rest der Platte
soll dann zur Partition f�rs LINUX Dateisystem werden.

 

\section{Die Installation von LINUX}

Dazu benutzen wir das Programm \eing{setup}. Als erstes wird unter
der Option KEYMAP die deutsche Tastatur eingestellt.
Dazu wird folgende Zeile: \eing{/usr/lib/kbd/keytables/gr-lat1.map.gz}
ausgew�hlt.

Danach will \prog{setup} die SWAP-Partition installieren. Das f�hren wir
durch. Danach wird \prog{mkswap} und \prog{swapon} durchgef�hrt.

Wir setzten dann die eigentliche Installation von Diskette fort.
Es wird auf der LINUX-Partition installiert, und dort wird als erstes
ein \eing{ext2} filesystem installiert. Danach wird die LINUX-Partition 
formatiert.

Die DOS-Partition soll unter dem Namen \eing{/dosc} angesprochen werden
k�nnen.

F�r das Praktikum werden folgende Diskettens�tze installiert:

\begin{verbatim}
  | |    [X] CUS  Also prompt for CUSTOM disk sets                        | |
  | |    [X] A    Base Linux system                                       | |
  | |    [X] AP   Various Applications that do not need X                 | |
  | |    [X] D    Program Development (C, C++, Lisp, Perl, etc.)          | |
  | |    [ ] E    GNU Emacs                                               | |
  | |    [ ] F    FAQ lists, HOWTO documentation                          | |
  | |    [ ] I    Info files readable with info or Emacs                  | |
  | |    [ ] IV   InterViews Development + Doc and Idraw Apps for X       | |
  | |    [ ] N    Networking (TCP/IP, UUCP, Mail, News)                   | |
  | |    [ ] OI   ObjectBuilder for X                                     | |
  | |    [ ] OOP  Object Oriented Programming (GNU Smalltalk 1.1.1)       | |
  | |    [ ] Q    Replacement Linux kernels with special drivers          | |
  | |    [ ] T    TeX                                                     | |
  | |    [ ] TCL  Tcl/Tk/TclX, Tcl language, and Tk toolkit for X         | |
  | |    [X] X    Base XFree-86 2.0 X Window System                       | |
  | |    [X] XAP  X Applications                                          | |
  | |    [X] XD   XFree-86 2.0 X11 Program Development System             | |
  | |    [X] XV   XView 3.2 release 5. (OpenLook Window Manager, apps)    | |
  | |    [ ] Y    Games (that do not require X)                           | |
\end{verbatim}

Unsere Custom Disketten hei�en \eing{feu}. Die weiteren Einstellungen
(time zone, Maus, CD-ROM, ...) d�rften klar sein.

Da wir den Studenten, die auf Ihrem Heim-PC LINUX installieren,
empfehlen LINUX von Diskette zu booten, habe ich keine Erfahrung,
wie man LILO direkt mit dem \prog{setup}-Programm installiert.
Aber das sollte eingentlich aus der Benutzerf�hrung von
\prog{setup} hervorgehen. Neben LILO sollten wir trotzdem eine
Bootdiskette mit LINUX f�r den Notfall erstellen. Ich hoffe, da�
das Rechenzentrum auch daf�r noch ein paar 5,25'' Disketten �brig
hat. Siehe Abschnitt 7!

Das \prog{selection}-Programm brauchen wir nicht, es wird also 
nicht installiert.

Nachdem die Installation soweit abgeschlossen ist, wird der Rechner neu
gebootet, und zwar mit der gerade erstellten Bootdiskette.



\section{Einrichten eines Benutzers}
F�r den Studenten soll gleich ein Benutzer eingerichtet werden.
Benutzernummer, sowie Passwort soll der Nachname des Studenten sein.

Group-ID und User-ID sind egal und k�nnen von \prog{adduser} selbst
bestimmt werden. Das Home-Directory soll wie vorgeschlagen unter
\eing{/home/Nachname} stehen. Die Default-Shell soll die \eing{bash} sein.

\section{Konfiguration von LILO}
Um sp�ter wahlweise Linux oder MS-Dos booten zu k�nnen, mu� nun noch die Datei
\prog{/etc/lilo.conf} editiert werden.

Die Datei sollte danach folgenden Inhalt haben:
\begin{verbatim}
boot = /dev/hda
vga = normal
ramdisk = 0
prompt
root = /dev/hda3
image = /vmlinuz
label = linux
other = /dev/hda1
label = msdos
\end{verbatim}

Nach Abschlu� der �nderungen nochmals \eing{lilo} aufrufen.
\end{document}
