\documentstyle[german,din-a4,listing]{article}
%\makeindex
\pagestyle{headings}

\title{XGraphic-Bibliothek}
\author{Alexander Sch"olling \and
Stefan Wohlfeil}
\date{M"arz 1994; aktualisiert im August 1995}

\begin{document}
% Auf das Inhaltsverzeichnis verzichten wir besser.
% 15.3 Stefan
% \tableofcontents
% \pagebreak
\maketitle





\section{Einf"uhrung}

 Die Programmierung unter einem graphischen Fenstersystem (wie z.B. 
 X-Windows) unterscheidet sich grundlegend von der Programmierung auf
 zeilenorientierten Terminals.
 Dort hat in der Regel das Programm den Benutzer gef"uhrt. Brauchte
 das Programm verschiedene Eingaben vom Benutzer, so hat es diese
 nacheinander, z.B. durch Men"us abgefragt. Ein Benutzer konnte also
 immer nur eine bestimmte Eingabe machen.

 Ein typisches Programmst"uck sah wie folgt aus:
\begin{verbatim}
 REPEAT
    ZeigeMenue1();
    Taste := LeseTastatur()
 UNTIL ErlaubtInMenue1 (Taste);
 REPEAT
    ZeigeMenue2();
    Taste := LeseTastatur()
 UNTIL ErlaubtInMenue2 (Taste);
 ...
\end{verbatim}

 Heutzutage m"ochte ein Benutzer sich aber nicht vom Programm vorschreiben
 lassen, wann er welche Eingaben machen mu"s. Das Programm soll auf die
 Eingaben des Benutzers reagieren. Aus Sicht des Programms treten also
 st"andig Ereignisse auf, auf die das Programm reagieren mu"s. M"ogliche
 Ereignisse sind:

\begin{itemize}
   \item Taste gedr"uckt bzw. losgelassen
   \item rechter/mittlerer/linker Mausknopf gedr"uckt bzw. losgelassen
   \item Mauszeiger ist in einen bestimmten Bildschirmbereich eingetreten
   \item ...
\end{itemize}

 Au"serdem hat man heute in der Regel mehrere Programme gleichzeitig
 auf einem Computer laufen. Daher m"ussen die auftretenden Ereignisse
 dem richtigen Programm zugeordnet werden. Dr"uckt der Benutzer z.B.
 eine Taste, so mu"s entschieden werden welches Programm auf diesen
 Tastendruck reagieren soll.
 
 Ein typisches Programmst"uck sah wie folgt aus:
\begin{verbatim}
 REPEAT
    Ereignis := HoleEreignis();
    BearbeiteEreignis (Ereignis);
 UNTIL fertig;

 PROCEDURE BearbeiteEreignis ( Ereignis : ... );
 BEGIN
    CASE Ereignis OF

       (* lange Liste aller moeglichen Ereignisse *)

    END; (* CASE *)
 END; (* BearbeiteEreignis *)
\end{verbatim}

 Um sich die Programmierung zu erleichtern gibt es nun verschiedene
 Bibliotheken, die einige, immer gleich zu behandelnde Ereignisse
 bearbeiten und ansonsten die Ereignisse automatisch an das richtige
 Programm weiterleiten. Als Programmierer mu"s man nur noch angeben,
 welche der selbstgeschriebenen Funktionen bei einem bestimmten
 Ereignis aufgerufen werden soll.

 Der typische Ablauf dabei ist wie folgt: Zun"achst erzeugt der
 Programmierer die Objekte (Fenster, Menues, Scrollbalken, Kn"opfe, etc.)
 aus denen die Oberfl"ache bestehen soll. Anschliessend wird eine
 Prozedur aufgerufen, die die eintretenden Ereignisse "uberwacht und
 die jeweiligen Funktionen, die der Programmierer bei der Erzeugung
 der Objekte angegeben hat, aufruft.

 Ein typisches Programmst"uck sieht wie folgt aus:
\begin{verbatim}
 PROCEDURE BearbeiteKnopf ( ... (* Parameter siehe jeweils im Handbuch *) );
 BEGIN
    WriteString("Der Knopf wurde gedrueckt!");
 END;

 BEGIN
 ...
 Erzeuge_Objekt( Knopf, 
     KnopfText,  "Hier druecken!",
     KnopfKoordinaten,   100, 100,
     KnopfEventHandler,  BearbeiteKnopf, (* Hier wird als Parameter die   *)
                                         (* Adresse einer Prozedur ueber- *)
                                         (* geben.                        *)
     ENDE );

 Hauptschleife();
 END.
\end{verbatim}

 In der Prozedur Hauptschleife, die vom System zur Verf"ugung gestellt wird,
 werden alle auftretenden Events bearbeitet und bei Bedarf die vom
 Programmierer geschriebenen Funktionen (BearbeiteKnopf) aufgerufen.
 Mehr zu diesem Thema k"onnen Sie u.a. im Kurs 1817 {\em Programmierung
 graphischer Benutzerschnittstellen} nachlesen.

 Welche Objekte vorhanden sind, welche Attribute (z.B. Koordinaten) ein
 bestimmtes Objekt haben kann, wie die Parameterliste der zu
 programmierenden Call-back Funktionen (BearbeiteKnopf) genau aussieht
 und viele weitere Details mu"s der Programmierer den Handb"uchern der
 Bibliothek entnehmen. Der typische Umfang dieser Handb"ucher liegt
 bei ca. 600 Seiten. Damit Sie sich im Praktikum nun nicht mit allen
 Details befassen m"ussen, habe wir eine kleine Bibliothek mit den
 wichtigsten Funktionen f"ur Sie erstellt.
 


%
% Noch einen Abschnitt zu Toolkits wie xview einfuegen!
% Werde ich (Stefan) noch schreiben.
%
% 14.3.94 Erledigt.
%

Die Bibliothek XGraphic erleichtert Ihnen die Programmierung von
unter X Windows ablaufenden Programmen, indem sie
etliche Routinearbeiten bereits
erledigt. Dazu geh"ort zum Beispiel der vordefinierte Button QUIT mit der
dazugeh"origen Funktion; dadurch brauchen Sie sich nicht mehr darum k"ummern,
da"s ein Benutzer Ihr Programm beenden kann. Auch wird die Programmierung von
Men"us auf die Verwendung von drei Befehlen mit wenigen Parametern
reduziert. Die Bibliothek stellt Ihnen weiterhin einfache Befehle zur
Verf"ugung, um herauszufinden, wo sich die Maus befindet oder welche Taste
gedr"uckt wurde. In Kapitel \ref{Beispiel} sind zwei Beispielprogramme
abgedruckt. Das Erste ist das k"urzestm"ogliche Beispielprogramm, es tut nichts
au"ser ein Fenster auf den Bildschirm bringen. In diesem Fenster passiert
aber nichts. Das zweite Beispielprogramm erledigt bereits eine einfache
Benutzerkommunikation mit zwei einfachen Men"us und einem einfachen Malteil.





\section{Namenskonventionen}
Alle Funktionsnamen aus der Library beginnen in jedem Fall mit den
Gro"sbuchstaben XG, gefolgt von entweder einem Unterstrich oder einem
weiteren Gro"sbuchstaben und einem Unterstrich. Dieses K"urzel gibt die
Zugeh"origkeit der Funktion zu einer der folgenden Kategorien an:
\begin{itemize}\index{Funktionsgruppen}
   \item[XG]{Allgemeine Funktionen, Initialisierung des Systems und
       Start der Hauptschleife. Siehe Kapitel: \ref{XG}.}
   \item[XGM]{Men"ufunktionen, Definition und "Anderung von
       Men"us. Siehe Kapitel: \ref{XGM}.}
   \item[XGF]{Fu"szeilen-Funktionen, Textausgabe in die
       Fu"szeile des Fensters. Siehe Kapitel: \ref{XGF}.}
   \item[XGN]{Ein-/Ausgabe-Funktionen, Funktionen f"ur
       Mitteilungsfenster und Eingabefenster. Siehe Kapitel: \ref{XGN}.}
   \item[XGD]{Malfunktionen, Funktionen zum Zeichnen
       und Schreiben in den Malbereich des Fensters. Siehe Kapitel: \ref{XGD}.}
   \item[XGE]{Event-Behandlung, Funktionen zur Aufschl"u"sellung
       von Events. Kapitel \ref{XGE}.}
\end{itemize}





\section{Funktionsdefinitionen}
In diesem Abschnitt stellen wir Ihnen die einzelnen Funktionen der
Library vor. Zu jeder Funktion geben wir eine Erkl"arung zu den
Parametern der Funktion, sowie eine kurze Beschreibung was die
Funktion macht. Die genauen Namen der Funktion, die Typen der
Parameter, die Reihenfolge der Parameter, etc. entnehmen Sie dem
Definitionsmodul zu unserer Library, das Sie unter dem Pfad
\begin{verbatim}
/usr/local/mocka/lib/XGraphic.md
\end{verbatim}
finden k"onnen.

%
% evtl. einen Verweis einfuegen, wo der Definition Module zu
% finden ist. Das muss nicht hier im Dokument sein.
%
% abgehakt, Alexander, 9.3.94


\subsection{Allgemeine Funktionen\label{XG}}


\subsubsection{\underline{Init}\index{Init}}

\paragraph{Syntax}

\begin{verbatim}
XG_Init (AnzahlParameter, ParameterListe, PROC(RepaintCallback), PROC(EventCallback));
\end{verbatim}

\paragraph{Parameter}
%
% Das Wort Parameter steht schon vor dem Paragraphen. Also sollte
% es darin nicht unnoetig wiederholt werden.
% Daraufhin sollten alle Parameterbeschreibungen noch einmal
% durchgesehen werden.
%
"Ubergeben werden die Anzahl und eine Tabelle der Kommandozeilen-Parameter
sowie die Addressen der beiden ben"otigten Callback-Funktionen.


\paragraph{Funktionalit"at}
Das Programmfenster wird auf den Bildschirm gebracht und initialisiert.


\subsubsection{\underline{StartMainLoop}\index{StartmainLoop}}


\paragraph{Syntax}

\begin{verbatim}
XG_StartMainLoop();
\end{verbatim}

\paragraph{Parameter}
Keine.

\paragraph{Funktionalit"at}
Die Abarbeitung des sogenannten Mainloops wird gestartet. Von diesem Moment
an liegt die Ausf"uhrung beim Event-Handler und den Callbacks der Men"us.
Das Programm wird erst dann weiter ausgef"uhrt, wenn ein Event vorliegt.



\subsection{Men"u-Definition\label{XGM}}


\subsubsection{\underline{NewMenu}\index{NewMenu}}

\paragraph{Syntax}
\begin{verbatim}
IntVar := XGM_NewMenu (LabelString);
\end{verbatim}


\paragraph{Parameter}
Eine Zeichenkette (Men"u-Label).


\paragraph{Funktionalit"at}
Einleiten der Definition eines neuen Men"us. Es wird ein Handle auf das Men"u
zur"uckgeliefert, mit dessen Hilfe sp"ater im Programm einzelne Men"upunkte
deaktiviert und wieder aktiviert werden k"onnen.

\subsubsection{\underline{AppendItem}\index{AppendItem}}

\paragraph{Syntax}
\begin{verbatim}
XGM_AppendItem (MenuHandle, ItemString);
\end{verbatim}

\paragraph{Parameter}
Eine Zeichenkette sowie der Handle aus dem Aufruf von NewMenu.

\paragraph{Funktionalit"at}
Ein Men"upunkt wird an das Men"u angeh"angt.

\subsubsection{\underline{CreateMenu}\index{CreateMenu}}


\paragraph{Syntax}
\begin{verbatim}
XGN_CreateMenu (HMenuHandle, PROC(MenuCallbackProc))
\end{verbatim}

\paragraph{Parameter}
Addresse der Men"u-Callback-Funktion.
Handle des Men"us.

\paragraph{Funktionalit"at}
Diese Funktion schlie"st die Definition eines Men"us ab. Erst mit diesem
Aufruf wird das Men"u wirklich erzeugt und in das Fenster eingebaut!


\subsubsection{\underline{SetItemActivity}}

\paragraph{Syntax}
\begin{verbatim}
XGM_SetItemActivity (MenuHandle, ItemNumber, StatusFlag);
\end{verbatim}

\paragraph{Parameter}
"Ubergeben werden der Men"uhandle, die laufende Nummer des zu bearbeitenden
Eintrags (Nummerierung beginnt mit eins) sowie der neue Status (Bool'scher Wert,
TRUE entspricht ausw"ahlbar, FALSE inaktiv).

\paragraph{Funktionalit"at}
Mit SetItemActivity k"onnen einzelne Punkte von Men"us desaktiviert bzw.
reaktiviert werden.


\subsection{Fu"szeilen bearbeiten\label{XGF}}


\subsubsection{\underline{SetLeft}\index{SetLeft}}

\paragraph{Syntax}
\begin{verbatim}
XGF_SetLeft (StringConst);
\end{verbatim}

\paragraph{Parameter}
Eine Zeichenkette (anzuzeigender String).

\paragraph{Funktionalit"at}
Der "ubergebene String wird in der linken Fu"szeile angezeigt und
"uberschreibt damit den bisher dort angezeigten Text.


\subsubsection{\underline{SetRight}\index{SetRight}}

\paragraph{Syntax}
\begin{verbatim}
XGM_SetRight (StringConst);
\end{verbatim}

\paragraph{Parameter}
Eine Zeichenkette (anzuzeigender String).

\paragraph{Funktionalit"at}
Der "ubergebene String wird in der rechten Fu"szeile angezeigt und
"uberschreibt damit den bisher dort angezeigten Text.


\subsubsection{\underline{SetDefault}\index{SetDefault}}

\paragraph{Syntax}
\begin{verbatim}
XGM_SetDefault();
\end{verbatim}

\paragraph{Parameter}
Keine.

\paragraph{Funktionalit"at}
Der Text in den Fu"szeilen wird durch den Standard-Text ersetzt
(Links: Copyright-Meldung; Rechts: Versions-Nummer).




\subsection{Ein-/Ausgabe-Fenster\label{XGN}}


\subsubsection{\underline{InputInt}\index{InputInt}}

\paragraph{Syntax}
\begin{verbatim}
XGN_InputInt (PromptString, InitialValue, MinValue, MaxValue, NumberOfTicks);
\end{verbatim}

\paragraph{Parameter}
Eine Zeichenkette (Prompt), Vorgabewert, Untergrenze, Obergrenze, Anzahl Unterteilungen.

\paragraph{Funktionalit"at}
Es wird ein Fenster angezeigt, in dem der Benutzer durch Verschieben
des Sliders eine Zahl im angegebenen Bereich eingeben kann. Es werden
keine Events generiert, bevor der Benutzer eine Zahl eingegeben hat.
Das Fenster enth"alt zwei Buttons, die mit OK und CANCEL beschriftet
sind. Egal welcher Button gedr"uckt wird, mit der Funktion GetLastInteger
wird in jedem Fall der gew"ahlte Wert geliefert. Weitere Erl"auterungen
unter der Funktion GetlastInteger.


\subsubsection{\underline{GetlastInteger}\index{GetLastInteger}}

\paragraph{Syntax}
\begin{verbatim}
BoolVar := XGN_GetLastInteger (IntVar);
\end{verbatim}

\paragraph{Parameter}
Eine INTEGER-Variable.


\paragraph{Funktionalit"at}
Mit dieser Funktion kann der letzte eingegebene Wert erfragt werden.
Der R"uckgabewert ist TRUE, wenn der Benutzer den OK-Button dr"uckte,
FALSE sonst.

\subsubsection{\underline{InputString}\index{InputString}}

\paragraph{Syntax}
\begin{verbatim}
XGN_InputString (LabelString);
\end{verbatim}

\paragraph{Parameter}
Eine Zeichenkette (Prompt).

\paragraph{Funktionalit"at}
Wie auch InputInt bringt InputString ein Fenster auf den Bildschirm, in
dem der Benutzer einen Text eingeben kann. Die Generierung von Events
ist blockiert, bis der Benutzer einen der beiden Buttons anklickt.


\subsubsection{\underline{GetLastString}\index{GetlastString}}

\paragraph{Syntax}
\begin{verbatim}
BoolVar := XGN_GetLastString (StringVar);
\end{verbatim}

\paragraph{Parameter}
Eine String-Variable.


\paragraph{Funktionalit"at}
Wenn der Benutzer die Eingabe durch den Button OK abgeschlo"sen hat, so
wird TRUE als R"uckgabewert geliefert. Wurde der CANCEL-Button
angeklickt, so wird FALSE zur"uckgegeben und als Text wird "CANCELLED"
geliefert.


\subsubsection{\underline{ShowNotice}\index{ShowNotice}}

\paragraph{Syntax}
\begin{verbatim}
XGN_ShowNotice (StringConst);
\end{verbatim}

\paragraph{Parameter}
Eine Zeichenkette (Auszugebender Text).


\paragraph{Funktionalit"at}
Die Ausf"uhrung des Programms wird solange angehalten, bis der Benutzer
"uber den OK-Button die Anzeige quittiert hat. Der Text kann Newline-Zeichen
(ASCII 13) enthalten, um einen Zeilenumbruch des Textes zu erzwingen.



\subsubsection{\underline{FileSelector}}

\paragraph{Syntax}
\begin{verbatim}
XGN_FileSelector (MaskString);
\end{verbatim}

\paragraph{Parameter}
Eine Zeichenkette (regul"arer Ausdruck), die die anzuzeigenden Dateien angibt.

\paragraph{Funktionalit"at}
Es wird eine Fileselector-Box auf den Bildschirm gebracht, die den Inhalt des
aktuellen Verzeichnisses alphabetisch sortiert (Verzeichnisse vor Dateien)
anzeigt.

\subsubsection{\underline{GetFileName}}

\paragraph{Syntax}
\begin{verbatim}
BoolVar := XGN_GetFileName (StringVar);
\end{verbatim}

\paragraph{Parameter}
Eine Stringvariable.

\paragraph{Funktionalit"at}
Wie auch bei GetLastInteger und GetLastString wird hier der zuletzt eingegebene
Wert (Dateiname inkl. Pfad) sowie der gedr"uckte Button (TRUE entspricht OPEN)
geliefert.

\subsubsection{\underline{YesOrNo}}

\paragraph{Syntax}
\begin{verbatim}
BoolVar := XGN_YesOrNo (StringConst);
\end{verbatim}

\paragraph{Parameter}
Zeichenkette.

\paragraph{Funktionalit"at}
Diese Funktion gibt eine Notice aus (siehe ShowNotice), die jedoch zwei Buttons
enth"alt (YES und NO). TRUE wird geliefert, wenn YES gedr"uckt wurde, FALSE sonst.

\subsection{Malfunktionen\label{XGD}}
%
% Hier koennten wir auch kurz etwas zum verwendeten Koordinatensystem
% sagen. Wo liegt (0,0) und in welche Richtung zeigen steigende x bzw.
% y Werte? Welches ist der maximale x bzw. y Wert?
%
Der Wertebereich der Koordinaten liegt zwischen $(0, 0)$ und $(9999, 9999)$,
der Ursprung liegt in der oberen linken Ecke. Bei allen Funktionen wird zuerst
die X-Koordinate angegeben. Die Farben sind als Aufz"ahlungstyp im Definitions-
Modul abgelegt.

\subsubsection{\underline{BeginPoly}}

\paragraph{Syntax}
\begin{verbatim}
XGD_BeginPoly (MaxN);
\end{verbatim}

\paragraph{Parameter}
"Ubergeben wird die gew"unschte Maximalzahl von Polygonpunkten.

\paragraph{Funktionalit"at}
Leitet die Definition eines zu malenden Polygons oder Linienzuges ein.

\subsubsection{\underline{AddPolyPoint}}

\paragraph{Syntax}
\begin{verbatim}
XGD_AddPolyPoint (px, py);
\end{verbatim}

\paragraph{Parameter}
Die Koordinaten des neuen Polygonpunktes. Wenn mehr Aufrufe erfolgen, als
Polygonpunkte bei BeginPoly angegeben waren, so werden die zus"atzlichen
Punkte ignoriert. Um ein Polygon zu schliessen, wird {\em nicht} der Startpunkt
nocheinmal als letzter Punkt angegeben!

\paragraph{Funktionalit"at}
Anf"ugen eines Punktes an das mit BeginPoly eingeleitete Polygon.

\subsubsection{\underline{DrawPoly}}

\paragraph{Syntax}
\begin{verbatim}
XGD_DrawPoly (Closed, Filled, Width, Color);
\end{verbatim}

\paragraph{Parameter}
Zwei Bool'sche Werte (Closed und Filled), die angeben, ob das Polygon
geschlo"sen bzw. gef"ullt ist. Die Breite, in der die Linien gezeichnet
werden sowie die Farbe.

\paragraph{Funktionalit"at}
Schlie"st die Definition eines Polygons oder eines Linienzuges ab und malt ihn.
Wenn $Closed = TRUE$, so wird eine Linie vom letzten zum ersten Punkt
gezeichnet. Wenn $Filled = TRUE$, so wird das Polygon in der angegebenen Farbe
gef"ullt. Es wird dann nat"urlich auch geschlossen.


\subsubsection{\underline{DrawLine}\index{DrawLine}}

\paragraph{Syntax}
\begin{verbatim}
XGD_DrawLine (StartX, StartY, EndX, EndY, Breite, Farbe);
\end{verbatim}

\paragraph{Parameter}
F"unf ganze Zahlen (Anfangs- und Endkoordinaten, Liniendicke).
Element von tColor (Farbe).


\paragraph{Funktionalit"at}
Die Funktion zeichnet eine Linie vom Start- zum Endpunkt in der angegebenen
Dicke und Farbe.


\subsubsection{\underline{DrawCircle}\index{DrawCircle}}

\paragraph{Syntax}
\begin{verbatim}
XGD_DrawCircle (CenterX, CenterY, Radius, Filled, Color);
\end{verbatim}

\paragraph{Parameter}
Vier ganze Zahlen (Mittelpunkts-Koordinaten, Radius, Farbe).
Ein Wahrheitswert, der angibt, ob der Kreis gef"ullft werden soll.

\paragraph{Funktionalit"at}
Die Funktion zeichnet einen Kreis mit dem gew"unschten Radius um den
angegebenen Mittelpunkt in der angegebenen Farbe. Wenn Filled TRUE ist,
so wird der Kreis in der entsprechenden Farbe ausgef"ullt.


\subsubsection{\underline{DrawString}\index{DrawString}}

\paragraph{Syntax}
\begin{verbatim}
XGD_DrawString (PosX, PosY, Text, Color);
\end{verbatim}

\paragraph{Parameter}
Zwei ganze Zahlen (Koordinaten), die Farbe und eine Zeichenkette.

\paragraph{Funktionalit"at}
Die "ubergebene Zeichenkette wird in der angegebenen Farbe ausgegeben.
Die "ubergebenen Koordinaten bezeichnen die linke untere Ecke der Bounding
Box!


\subsubsection{\underline{Clear}\index{Clear}}

\paragraph{Syntax}
\begin{verbatim}
XGD_Clear();
\end{verbatim}

\paragraph{Parameter}
Keine.

\paragraph{Funktionalit"at}
Der Malbereich des Programmfensters wird gel"oscht.

\subsubsection{\underline{DrawArrowLine}}

\paragraph{Syntax}
\begin{verbatim}
XGD_DrawArrowLine(StartX, StartY, EndX, EndY, Width, Color);
\end{verbatim}

\paragraph{Parameter}
Erl"auterungen siehe DrawLine.

\paragraph{Funktionalit"at}
Es wird eine Linie zwischen den angegebenen Punkten gemalt. Bei 60 Prozent
der L"ange befindet sich eine Pfeilspitze.

\subsubsection{\underline{Flush}}

\paragraph{Syntax}
\begin{verbatim}
XGD_Flush();
\end{verbatim}

\paragraph{Parameter}
Keine.

\paragraph{Funktionalit"at}
Der Windowmanager wird gezwungen, ein Repaint durchzuf"uhren (wird ben"otigt,
wenn z.B. als Folge eines animierten Ablaufs eines Malvorgangs ein Repaint
ben"otigt w"urde, jedoch nicht automatisch ausgef"uhrt wird.

\subsection{Event-Behandlung\label{XGE}}


\subsubsection{\underline{Button\_L}\index{Button\_L}}

\paragraph{Syntax}
\begin{verbatim}
BoolVar := XGE_Button_L();
\end{verbatim}

\paragraph{Parameter}
Keine.

\paragraph{Funktionalit"at}
Diese Funktion gibt zur"uck, ob der Event durch die linke Maustaste
ausgel"ost wurde (R"uckgabewert TRUE) oder nicht.


\subsubsection{\underline{Button\_M}\index{Button\_M}}

\paragraph{Syntax}
\begin{verbatim}
BoolVar := XGE_Button_M();
\end{verbatim}

\paragraph{Parameter}
Keine.

\paragraph{Funktionalit"at}
Diese Funktion arbeitet wie Button\_L, gibt jedoch an, ob die mittlere
Maustaste den Event ausgel"ost hat.


\subsubsection{\underline{Button\_R}\index{Button\_R}}

\paragraph{Syntax}
\begin{verbatim}
BoolVar := XGE_Button_R();
\end{verbatim}

\paragraph{Parameter}
Keine.

\paragraph{Funktionalit"at}
Wie Button\_L, jedoch f"ur die rechte Maustaste.


\subsubsection{\underline{X\_Position}\index{Position\_X}}

\paragraph{Syntax}
\begin{verbatim}
IntVar := XGE_X_Position();
\end{verbatim}

\paragraph{Parameter}
Keine.

\paragraph{Funktionalit"at}
Die X-Koordinate der Maus zum Zeitpunkt des Eintretens des Events.


\subsubsection{\underline{Y\_Position}\index{Y\_Position}}

\paragraph{Syntax}
\begin{verbatim}
IntVar := XGE_Y_Position();
\end{verbatim}

\paragraph{Parameter}
Keine.

\paragraph{Funktionalit"at}
Wie X\_Position, jedoch f"ur die Y-Koordinate.


\subsubsection{\underline{KeyCode}\index{KeyCode}}


\paragraph{Syntax}
\begin{verbatim}
CharVar := XGE_KeyCode();
\end{verbatim}
\paragraph{Parameter}
Keine.

\paragraph{Funktionalit"at}
Als R"uckgabewert wird der ASCII-Code des Zeichens geliefert, das
den aktuellen Event ausgel"ost hat, beziehungsweise das Null-Zeichen,
wenn keine Taste f"ur den Event verantwortlich war.




\section{Beispiele\label{Beispiel}}




\subsection{Einfachstes Programm}
\begin{verbatim}
MODULE SimpleX;

FROM Arguments IMPORT GetArgs, ArgTable;

FROM XGraphic IMPORT XG_Init, XG_StartMainLoop;

PROCEDURE Event() : BOOLEAN;
BEGIN
(* Hier wuerden normalerweise Events aufgeschluesselt und die
   entsprechenden Aktionen eingeleitet. *)
    RETURN FALSE;
END Event;

PROCEDURE Repaint();
BEGIN
(* Normalerweise wuerde hier der Fensterinhalt neu gemalt *)
END Repaint;

VAR Anz : SHORTINT; (* Hier merken wir uns die Anzahl der Kommandozeilen-
                       Parameter *)
    Args : ArgTable; (* Und hier die Parameter *)

BEGIN
    GetArgs (Anz, Args);
    (* Holen der Parameter *)
    XG_Init (Anz, Args, PROC(Repaint), PROC(Event));
    XG_StartmainLoop();
END SimpleX.
\end{verbatim}

Dieses einfache Beispielprogramm bringt lediglich ein Fenster auf den
Bildschirm, in dem nicht gemalt wird. Die Funktionen {\em Repaint} und
{\em Event}
sind leere Funktionen, die jedoch trotzdem vorhanden sein m"ussen. Durch
den Befehl {\em PROC(Repaint)} wird erreicht, da"s an dieser Stelle nicht die
Funktion Repaint aufgerufen, sondern nur ihre Addresse eingesetzt wird.
(Analog f"ur Event.)




\subsection{Ein Gr"o"seres Beispiel}
\begin{verbatim}
MODULE GreaterX;

FROM Arguments IMPORT GetArgs, ArgTable;

FROM XGraphic IMPORT XG_Init, XG_StartMainLoop,
                     XGN_ShowNotice,
                     XGM_NewMenu, XGM_AddItem, XGM_CreateMenu,
                     XGD_Clear, XGD_DrawCircle,
                     XGE_X_Position, XGE_Y_Position,
                     XGE_Button_L, XGE_Button_R;

VAR cx, cy : INTEGER;
    Anz : SHORTINT;
    Args : ArgTable;
    handle : INTEGER;

PROCEDURE Repaint();
BEGIN
    XGD_Clear();
    XGD_DrawCircle (cx, cy, 10, FALSE, 0);
    (* Male einen Kreis um den Punkt (cx, cy) mit dem Radius 10 in der Farbe
       Schwarz (nicht ausgefuellt) *)
END Repaint;

PROCEDURE Event() : BOOLEAN;
BEGIN
    IF XGE_Button_L=1 THEN
        cx := XGE_X_Position();
        cy := XGE_Y_Position();
        RETURN TRUE;
    END;
    RETURN FALSE;
END Event;
(* Beachte: Nach Beendigung der Event-Funktion wird vom System die
Repaint-Funktion aufgerufen, sofern dies notwendig ist (RETURN TRUE). *)


PROCEDURE Menue (Nummer : INTEGER);
BEGIN
    IF (Nummer = 2)
        XGN_ShowNotice ("Hello, World");
    END;
END Menue;

BEGIN
    GetArgs (Anz, Args);
    XG_Init (Anz, Args, PROC(Repaint), PROC(Event));
    handle := XGM_NewMenu ("Testmenue");
    XGM_AddItem (handle, "Punkt 1");
    XGM_AddItem (handle, "Hallo!");
    XGM_CreateMenu (handle, PROC(Menue));
    XG_StartMainLoop();
END GreaterX.
\end{verbatim}

In diesem Beispiel wird zus"atzlich ein Men"u eingef"ugt, das bei Auswahl des
zweiten Men"u-Punktes eine Nachricht auf den Bildschirm bringt. Au"serdem
reagiert das Programm auf Mausclicks innerhalb des Malbereichs (der gro"sen
Fl"ache unterhalb der Men"uzeile), indem es um den Punkt, an dem die linke
Maustaste gedr"uckt wurde, einen Kreis zeichnet.
    
%
% Kurze Erklaerung zu diesem Beispiel
%
% erledigt - Alexander





\section{Systemanforderungen}




\subsection{Hardware}
Ben"otigt wird dieselbe Ausstattung, die auch f"ur Linux erforderlich ist.
Man braucht also einen PC, der mindestens mit einem 80386SX ausgestattet ist.
Au"serdem m"ussen 8 MByte RAM und ca. 200 MByte Platz auf der Festplatte
vorhanden sein.
Zus"atzlich braucht man eine Maus mit drei Tasten (Mouse-Systems kompatibel),
%
% Vielleicht sollten wir noch sagen, dass XFree der Treiber fuer
% X ist?
%
sowie eine SVGA-Grafikkarte mit einem von XFree86 V2.0\footnote[1]{XFree86 ist
der X-Server f"ur Linux und als solcher sorgt er daf"ur, da"s Programme unter
X Windows laufen k"onnen} unterst"utzten Chipsatz.
%
% im Duden nachsehen ob Grafikkarte mit 1 oder 2 k geschrieben wird.
%
Von XFree86 werden zur Zeit unter anderem folgende Chips"atze
unterst"utzt: S3, Mach8, Mach32, ET4000 (und ET4000W32), Trident 8900, IBM 8514, Genoa. Wenn Sie eine genauere Liste haben wollen, so k"onnen Sie diese von
folgenden Quellen beziehen:
\begin{enumerate}
\item{Viele Buchhandlungen haben das Linux Anwenderhandbuch vorr"atig. Im
Kapitel zur Konfiguration von XFree86 werden die Unterst"utzten Chips"atze
aufgelistet.}
\item{Die Linux-CDs der Fa. SuSE listen auf der R"uckseite der H"ulle ebenfalls
die Chips"atze auf.}
\end{enumerate}




\subsection{Software}
Auf dem PC mu"s Linux in der Kernel Version 0.99pl13 oder h"oher,
XFree86 ab Version 1.3 (besser w"are Version 3.1x) und der Modula 2
Compiler MOCKA ab Version 9302 installiert sein.




\section{Installation}
%
% Diesen Abschnitt sollten wir in zwei Teile aufteilen. Zunaechst
% einen Paragraphen, in dem der allgemeine Ablauf der Installation
% beschrieben wird.
% Danach dann eine Schritt fuer Schritt anleitung, die dann wie folgt
% aussehen sollte:
% 1. Disketten einlegen
% 2. ...
% n. Hier unbedingt ja eingeben
%
% Hiermit geschehen :-) - Alexander



\subsection{Allgemeines}
Die Installation geschieht unter Verwendung der von uns erstellten
Programmdiskette, die neben dem Compiler und der Bibliothek auch die
notwendigen Files enth"alt, die vom Shellscript SETUP, das bei der von uns
verwendeten Slackware-Linux-Distribution enthalten ist, automatisch
installiert werden kann. Au"serdem ist eine Konfigurations-Datei f"ur
das Paket MTOOLS auf der Diskette enthalten; diese darf jedoch nur
installiert werden, wenn Ihr Laufwerk A: ein 5,25\"\/-Laufwerk ist.
Die n"achsten drei Abschnitte beschreiben die Installation Schritt f"ur
Schritt in Abh"angigkeit vom bei Ihnen installierten Linux-System. Benutzer
der Slackware-Distributionen vor Version 1.1.2 benutzen bitte Version A,
Benutzer neuerer Slackware-Distributionen (hierzu z"ahlen auch diejenigen von
Ihnen, die Ihr Linux von uns erhalten haben) verwenden bitte Version B. Alle
anderen m"ussen Version C benutzen.




\subsection{Schritt f"ur Schritt - Version A}
Um den Compiler und die Bibliothek zu installieren (vorausgesetzt, Sie
haben Linux mit der Slackware-Distribution in einer Version vor 1.1.2
installiert), gehen Sie bitte wie folgt vor:

\begin{enumerate}
  \item{Diskette einlegen}
  \item{Login als ROOT}
  \item{Eingabe: setup und RETURN}
  \item{Auswahl: Y (Neue Software installieren)}
  \item{Auswahl: 2 (Von Diskette installieren)}
  \item{Auswahl: 1 (Wenn die Diskette in Laufwerk A: ist, sonst 2 w"ahlen)}
  \item{Auswahl: 4 (Eine einzelne Diskette installieren)}
  \item{Auswahl: Y (Nachfrage bei jedem Paket)}
  \item{Eingabe: RETURN (Um die Installation zu starten)}
  \item{Eingabe: N oder Y (N, wenn Ihr Laufwerk A: ein 3,5\"\/-Laufwerk ist, Y
        sonst)}
  \item{Eingabe: N (Nicht neu konfigurieren)}
\end{enumerate}
Nun sind sowohl der MOCKA-Compiler als auch das XGraphic-Paket installiert.



\subsection{Schritt f"ur Schritt - Version B}
Um den Compiler und die Bibliothek mit dem Setup-Programm von Slackware 1.1.2
(das ist die von uns verschickte Version) zu installieren, gehen Sie bitte
wie folgt vor:

\begin{enumerate}
  \item{Diskette einlegen}
  \item{Login als ROOT}
  \item{Eingabe: setup und RETURN}
  \item{Auswahl: S und RETURN}
  \item{Auswahl: 2 und RETURN}
  \item{Auswahl des Laufwerks}
  \item{RETURN dr"ucken, um Disketten-Satz auszuw"ahlen}
  \item{Eingabe: LEERTASTE, dann RETURN}
  \item{Eingabe: P und dann RETURN}
  \item{RETURN dr"ucken}
  \item{RETURN dr"ucken}
  \item{Nach Abschlu"s der Installation das Setup-Programm beenden}
\end{enumerate}
Nun sind sowohl der MOCKA-Compiler als auch das XGraphic-Paket installiert.




\subsection{Schritt f"ur Schritt - Version C}

Wenn Sie nicht "uber das Setup-Programm der Slackware-Distribution verf"ugen
oder es aus irgendeinem Grund nicht benutzen wollen, so folgen Sie bitte den
folgenden Anweisungen:

\begin{enumerate}
   \item{Diskette einlegen}
   \item{Login als root}
   \item{Kopieren der Dateien in das tmp-Verzeichnis: mcopy \"\/a:*.tgz\"\/
/tmp}
   \item{Entpacken der Tar-Archive: cd /; tar xzf /tmp/mocka.tgz; tar xzf
/tmp/xgraphic.tgz}
\end{enumerate}
Nun sollten der Compiler und die Bibliothek installiert sein.










\section{Definitionsmodul}

\listing{XGraphic.md}

\end{document}
