\documentstyle[german,din-a4,time]{article}

%
% Hinter einem Prozentzeichen stehen Kommentare. Manchmal
% ist das ganz hilfreich. Der Text darf in dieser Datei uebrigens fast
% beliebig formatiert stehen, d.h. man darf einruecken, um die
% Lesbarkeit zu verbessern.
%

% Unter Linux hat Latex Probleme mit direkt eingegebenen Umlauten
% Fuer ae gibt man: "a ein
%      oe           "o
%      ue           "u
%      szett        "s

%
% Hier wird die Programmierbarkeit von LaTex benutzt. Es werden neue
% Kommandos definiert. (Alle haben einen Parameter)
%
\newcommand{\taste}[1]{[#1]}
\newcommand{\prog}[1]{{\bf #1}}
\newcommand{\eing}[1]{{\tt #1}}
\newcommand{\datei}[1]{{\tt #1}}

\begin{document}


\title{Wichtige Tips zum Umgang mit LINUX}
\author{Michael Fischer und Alexander Sch�lling}
\date{April 1994}
% \date{\today\ \thetime }
\maketitle





%
% Hier faengt ein Abschnitt an.
% Unterabschnitte heissen dann
% \subsection, \subsubsection, \paragraph, \subparagraph
%
\section{Einleitung}
Ein Rechner unter dem Betriebssystem LINUX wird anders benutzt und
bedient als ein Rechner unter DOS. Dieses Dokument ist ein Ausschnitt
aus der Installationsanleitung f�r die Studenten, die LINUX auf Ihrem
eigenen PC installieren wollen. Es enth�lt aber auch einige Informationen,
die f�r Studenten mit einem vorinstallierten Rechner wichtig sind. Lesen Sie
daher dieses Dokument sorgf�ltig, bevor Sie mit LINUX das
erste Mal arbeiten.
%
% Zwischen zwei Absaetze (Abschnitte) des Textes muss man eine Leerzeile
% eingeben, damit TeX erkennt, dass ein neuer Absatz beginnt.
% Es duerfen auch mehrere Leerzeilen sein, das ist egal
%




\section{Einschalten des Rechners}
Nach dem Einschalten des Rechners wird automatisch der Linux Bootmanager
gestartet.
Es erscheint nun ein Prompt, an dem man durch Dr�cken der
%
% hier wird die neue Funktion aufgerufen
%
\taste{Tab}-Taste eine Liste der verf�gbaren Betriebssysteme erh�lt. 
Um LINUX zu booten, gibt man an dieser Stelle \eing{linux} \taste{Return}
ein oder dr�ckt nur die \taste{Return}-Taste. Um MSDOS zu booten, gibt man 
\eing{msdos} ein und dr�ckt \taste{Return}.

Au�erdem k�nnen Sie den Rechner nat�rlich auch von einer Diskette booten.
Sie haben zusammen mit dem PC auch eine 5,25 Zoll Diskette bekommen.
Wenn Sie diese Bootdiskette beim Einschalten des Rechners in Laufwerk
A: legen, starten Sie auch das LINUX-System.




\section{Das Anmelden}


\subsection{Ihre Benuternummer}
Auf Ihrem PC ist bereits ein Benutzer f�r Sie eingerichtet. Wenn Sie sich
anmelden, m�ssen Sie als Login Ihren Nachnamen eingeben. Das Passwort
ist wieder Ihr Nachname. Normalerweise werden Sie unter dieser
Benutzerkennung arbeiten. Sie k�nnen so keinen Schaden im System anrichten.

Beachten Sie, da� unter UNIX in der Regel zwischen Gro�- und Kleinschrift
unterschieden wird! Au�erdem wird das Passwort, w�hrend Sie es eintippen,
%
% will man Woerter oder Satzteile hervorheben, so werden diese mit
% geschweiften Klammern umschlossen und darin steht das Kommando \em
%
{\em nicht} auf dem Bildschirm angezeigt.

Sie k�nnen jetzt Kommandos eingeben und z.B. durch Eingabe von
\eing{startx} das X-Window System starten.



\section{Kopieren der LINUX Bootdiskette}
Der n�chste Schritt besteht nun darin, eine Kopie Ihrer Bootdiskette
anzufertigen. Dazu ist es notwendig, zuerst den Inhalt der Bootdiskette 
auf die Festplatte zu kopieren. Zu diesem Zweck geben Sie das folgende 
Kommando ein:

%
% Alles zwischen begin{verbatim} und end{verbatim} wird im Zeichensatz
% Courir (Schreibmaschine) genauso gedruckt, wie es in der Datei steht.
%
\begin{verbatim}
darkstar:~# dd if=/dev/fd0 of=bootdisk
\end{verbatim}

Diese Eingabe kopiert alle Daten der Diskette in Ihrem Bootlaufwerk 
in ein File mit dem Namen \datei{bootdisk}. Dies dauert ein oder 
zwei Minuten. Danach nehmen Sie Ihre Bootdisk aus dem Bootlaufwerk 
heraus und legen eine andere formatierte Diskette in das Bootlaufwerk ein.
Um das File mit dem Inhalt Ihrer Bootdiskette auf die neue Diskette 
zu kopieren, geben Sie folgendes Kommando ein:

%
% Aufzaehlungen werden mit begin{itemize} und end{itemize} eingeklammert.
% Die Punkte der Aufzaehlung werden mit dem Kommando item eingeleitet.
% Ein Item darf aus mehreren Paragraphen bestehen.
% Ein Item darf wiederum eine Aufzaehlung enthalten
%
\begin{itemize}
\item Melden Sie sich zuerst ab.
\item Melden Sie sich dann als User root an.
\item geben Sie das Kommando \eing{halt} ein.
\item warten Sie bis die Meldung:
\end{itemize}



\section{Die Online-Hilfestellung}
LINUX stellt f�r alle Befehle eine Kurzbedienungsanleitung bereit, die sog. man-Page. Sie
k�nnen diese Kurzanleitungen jederzeit mit dem Kommando \prog{man} aufrufen. Sie
k�nnen z.B. Hilfestellung zum \prog{man}-Kommando bekommen, indem Sie
\eing{man man} eingeben.

Die man-Pages sind �brigens ins Deutsche �bersetzt worden und im LINUX-Anwenderhandbuch
abgedruckt. 

Welche Kommandos unter einem UNIX-System verf�gbar sind entnehmen Sie bitte der
Literatur �ber UNIX. Falls ein Kommando aus der Literatur nicht so arbeitet wie
dort beschrieben, so sehen Sie in den man pages nach wie dieses Kommando in unserem
System arbeitet.

Ansonsten k�nnen Sie auch mit dem Autor dieses Textes in Verbindung treten und
Ihre Fragen direkt stellen. Auch bei allen Problemen oder Abweichungen zu unserer
Anleitung sollten Sie sich bei uns melden. Sehr h�ufig k�nnen Sie die L�sung
aber auch selbst herausbekommen, indem Sie die Anleitungen aufmerksam durchlesen.
%
% Als Anfuehrungszeichen verwendet man zweimal das einfache Anfuehrungszeichen
%
In der Praxis ist das der ''normale'' Weg.


\end{document}
